% HINT: add option aspectratio=169 for 16:9 aspect ratio
% HINT: add 'handout' to activate handout mode (without overlays for printing)
\documentclass[t,english]{beamer}

% include package import and macro defintions
\input{includes/preamble}
\input{includes/rgb}
\input{includes/macros/general}
\input{includes/macros/commenting}

% Our own imports
\usepackage{listings, listings-rust}
\usepackage{pythonhighlight}

% add (multiple) bibliography sources for biblatex
\addbibresource{bib.bib}

% NOTE: the predefined options are best practices of AIM!
%       Do not change for seminar talks!
% Set authorinfo={0,1} to (de)activate short author and short title in footer
% Set progress={0,1} to switch between x and x/total display of pages in the bottom right corner
% Set outline={0,1} to (de)activate outline slides at the beginning of each section
\usetheme[authorinfo=1, progress=0, outline=0]{AIM}

% metadata
\title[Interim Presentation Group 2]{High-Performance Time Series Augmentation Library}
\subtitle{Interim Presentation}
\author[Basu Roy, Pradhan \& Kern]{Aryamaan Basu Roy \and Tejas Pradhan \and Felix Kern}
\date{\small\today}

% optional
\titlegraphic{
  \includegraphics[width=2cm]{figures/rwth-logo.png}
}

%%% NOTE: do not remove the following marker. We use it for automatic compilation of multiple versions
%%% of the slides (e.g., handout, presentation with overlays) via compile.py
%%%pythonmarker

\begin{document}

\begin{frame}[plain]
\titlepage
\end{frame}

% Start with some background on what the actual problem is

\begin{frame}{Introduction}
1. Time Series...What?
\begin{itemize}
        \item Finite Sequence
        \item Time ordered indexing
        \item Discrete Timestamps
        \item Overlapping/Non-Overlapping
    \end{itemize}
    
\end{frame}

\begin{frame}{Introduction}
1. Time Series...What?
\begin{itemize}
        \item Finite Sequence
        \item Time ordered indexing
        \item Discrete Timestamps
        \item Overlapping/Non-Overlapping
    \end{itemize}
    \vskip--15pt
2. Why Augment?
\begin{itemize}
        \item Improve Model Generalization
        \item Granular Data Analysis
        \item Simulate Real-World Noise
        \item Overlapping/Non-Overlapping
    \end{itemize}
    
\end{frame}

\begin{frame}{Insights From The Paper (\cite{wen2020time})}
1. Time Domain Transformations
\begin{itemize}
        \item Computationally Cheap
        \item Good baseline
    \end{itemize}
    
\end{frame}

\begin{frame}{Insights From The Paper (\cite{wen2020time})}
1. Time Domain Transformations
\begin{itemize}
        \item Computationally Cheap
        \item Good baseline
    \end{itemize}
 \vskip--15pt
2. Frequency Domain Transformations
\begin{itemize}
        \item Provides Global Structure 
        \item Great for Anomaly Detection 
    \end{itemize}
\end{frame}

\begin{frame}{Insights From The Paper (\cite{wen2020time})}
1. Time Domain Transformations
\begin{itemize}
        \item Computationally Cheap
        \item Good baseline
    \end{itemize}
 \vskip--15pt
2. Frequency Domain Transformations
\begin{itemize}
        \item Provides Global Structure 
        \item Great for Anomaly Detection 
    \end{itemize}
\vskip--15pt
3. Impact
\begin{itemize}
        \item 2\% Classification Accuracy Improvement 
        \item 20-30\% Improvement - Anomaly Detection F1 Score
        \item Reduced MASE error - Forecasting Use Cases
    \end{itemize}
\end{frame}

% Present project plan as given in the pdf gradually
\begin{frame}{Project Plan}
    1. Implementation of core time series augmentation methods in Rust
    \begin{itemize}
        \item Basic transformations
    \end{itemize}
\end{frame}

\begin{frame}{Basic Transformations (\cite{um2017data})}
    \textbf{Jittering}: Add white noise to the data
    \ig{1}{figures/timeseries/jittering.png}
    Parameters: $\sigma$ of Gaussian noise
\end{frame}

\begin{frame}{Basic Transformations (\cite{um2017data})}
    \textbf{Scaling}: Scale data points by random factor
    \ig{1}{figures/timeseries/scaling.png}
    Parameters: Min and max of random factor
\end{frame}

\begin{frame}{Basic Transformations (\cite{um2017data})}
    \textbf{Rotation}: "Rotate" data 180\textdegree\ around an anchor
    \only<1>{\ig{1}{figures/timeseries/rotation1.png}}
    \only<2>{\ig{1}{figures/timeseries/rotation2.png}}
    Parameters: Anchor value
\end{frame}

\begin{frame}{Basic Transformations (\refer{tsaug})}
    \textbf{Drop}: Drop percentage of data points
    \ig{1}{figures/timeseries/drop.png}
    Parameters: Percentage and default value
\end{frame}

\begin{frame}{Basic Transformations (\refer{tsaug})}
    Special Transformations:
    
    \vskip--15pt
    \twocol{\onslide<2-3>{\textbf{Repeat}: Repeat each series in the dataset $n$ times}}{\onslide<3>{\textbf{Crop}: Crop each time series into a random continuous slice of specified size}}
    
    \only<2>{\ig{0.8}{figures/timeseries/repeat.png}}
    \only<3>{\vskip--15pt \ig{0.8}{figures/timeseries/crop.png}}
    
\end{frame}

\begin{frame}{Project Plan}
    1. Implementation of core time series augmentation methods in Rust
    \begin{itemize}
        \item Basic transformations
        \item Time warping techniques
    \end{itemize}
\end{frame}


\begin{frame}{Time Warping}
    \textbf{Dynamic Time Warping (DTW):} Finds optimal warping path to generate different temporal variations.
    \ig{1}{figures/timeseries/DTW.png}
\end{frame}

\begin{frame}{Time Warping}
    \textbf{Window Warping:} Select a random window in the time series and stretch or compress it
    \ig{1}{figures/timeseries/window_warping.png}
\end{frame}

\begin{frame}{Project Plan}
    1. Implementation of core time series augmentation methods in Rust
    \begin{itemize}
        \item Basic transformations
        \item Time warping techniques
        \item Frequency-domain transformations
    \end{itemize}
\end{frame}

\begin{frame}{Frequency-Domain Transformations}
    \textbf{Fast Fourier Transform (FFT): } Converts a time-domain signal into its frequency components
    \begin{itemize}
        \item Time Domain
    \end{itemize}
    \ig{1}{figures/timeseries/fft_time.png}
\end{frame}

\begin{frame}{Frequency-Domain Transformations}
    \textbf{Fast Fourier Transform (FFT): } Converts a time-domain signal into its frequency components
    \begin{itemize}
        \item Frequency Domain
    \end{itemize}
    \ig{1}{figures/timeseries/fft_freq.png}
\end{frame}

\begin{frame}{Frequency-Domain Transformations}
    \textbf{Inverse Fast Fourier Transform (IFFT): } Converts frequency-domain data back to the time domain.
    \ig{1}{figures/timeseries/ifft.png}
\end{frame}

\begin{frame}{Frequency-Domain Transformations}
    \textbf{Inverse Fast Fourier Transform (IFFT): } Reconstruction error is negligible (below numerical threshold), ensuring lossless transform.
    \ig{1}{figures/timeseries/ifft_error.png}
\end{frame}

\begin{frame}{Frequency-Domain Transformations}
    \textbf{Frequency Masking: }Randomly zero out contiguous FFT bins around a center frequency to simulate narrowband interference or dropout in sensors.
    
    \ig{1}{figures/timeseries/masking.png}
\end{frame}

\begin{frame}{Frequency-Domain Transformations}
    \textbf{Amplitude \& Phase Perturbation (APP): }Add small Gaussian noise to each bin’s magnitude and phase to introduce realistic spectral jitter while preserving overall structure
    
    \ig{1}{figures/timeseries/APP.png}
\end{frame}

\begin{frame}{Project Plan}
    1. Implementation of core time series augmentation methods in Rust
    \begin{itemize}
        \item Basic transformations
        \item Time warping techniques
        \item Frequency-domain transformations
        \item Noise injection methods
    \end{itemize}
\end{frame}

\begin{frame}{Noise Injection}
    AddNoise augmenter that supports different kinds of noise (according to~\cite{wen2020time}):
    \begin{itemize}
        \item<2-> Uniform
        \item<3-> Gaussian (like jittering)
        \item<4-> Spike
        \item<5-> Slope
    \end{itemize}
    \only<2>{\ig{0.8}{figures/timeseries/uniform.png}}
    \only<3>{\ig{0.8}{figures/timeseries/jittering.png}}
    \only<4>{\ig{0.8}{figures/timeseries/spike.png}}
    \only<5>{\ig{0.8}{figures/timeseries/slope.png}}
\end{frame}

\begin{frame}{Project Plan}
    2. Develop a Python-friendly API that enables:
    \begin{itemize}
        \item<2-> Single-method augmentation calls
        \item<3-> Composable augmentation pipelines 
    \end{itemize}
\end{frame}

\begin{frame}[fragile]{Implementation - Rust}
    Every augmenter implements the Augmenter trait:
    \vskip--20pt
    \begin{lstlisting}[language=Rust, basicstyle=\small]
pub trait Augmenter {
    fn augment_dataset(&self, input: &mut Dataset) {
        input.features
            .iter_mut()
            .for_each(|x| self.augment_one(x));
    }
    
    fn augment_one(&self, x: &mut [f64]);
}
    \end{lstlisting}
\end{frame}

\begin{frame}[fragile]{Implementation - Rust}
    $\Rightarrow$ Allows us to create "recursive" augmenters:
    \vskip--20pt
    \begin{lstlisting}[language=Rust, basicstyle=\small]
pub struct ConditionalAugmenter {
    inner: Box<dyn Augmenter>,
    p: f64,
}

pub struct AugmentationPipeline {
    augmenters: Vec<Box<dyn Augmenter>>,
}
    \end{lstlisting}
\end{frame}

\begin{frame}[fragile]{Implementation - Rust}
    Example pipeline:
    \vskip--5pt
    \begin{lstlisting}[language=Rust, basicstyle=\small]
let pipeline = AugmentationPipeline::new() 
    + Repeat::new(10)
    + Crop::new(100)
    + ConditionalAugmenter::new(
        AddNoise::new(
            NoiseType::Slope, 
            Some((0.01, 0.02)), 
            None, 
            None
        ),
        0.25
    )
    + Jittering::new(0.1);
    
pipeline.augment_dataset(&mut dataset);
    \end{lstlisting}
\end{frame}

\begin{frame}{Implementation - Python}
    \begin{itemize}
        \item Python bindings using PyO3
        \item<2-> In seperate package to Rust library
        \item<3-> We bind the struct Dataset to pass the data around
        \item<4-> Augmenter structs are exposed to Python as classes
        \item<5-> Limitations with PyO3 don't allow for the same architecture
    \end{itemize}
\end{frame}

\begin{frame}[fragile]{Implementation - Python}
    PyO3 makes the binding of the recursive augmenters difficult, so these are fully written in Python for now:
    \vskip--5pt
    \begin{python}
class ConditionalAugmenter:
    def __init__(self, augmenter, probability): ...
    def augment_dataset(self, dataset: Dataset): ...
    def augment_one(self, x): ...
    
class AugmentationPipeline:
    def __init__(self): ...
    def __add__(self, other): ...
    def augment_dataset(self, dataset: Dataset): ...
    def augment_one(self, x): ...
    \end{python}
\end{frame}

\begin{frame}[fragile]{Implementation - Python}
    Full python example:
    \vskip--5pt
    \begin{python}
dataset = pf.Dataset(features, labels)

pipeline = (pf.AugmentationPipeline()
            + pf.Repeat(10)
            + pf.Crop(100)
            + pf.ConditionalAugmenter(
                pf.AddNoise(
                    pf.NoiseType.Slope, 
                    bounds=(0.01, 0.02)
                ),
                0.25
            )
            + pf.Jittering(0.1))

pipeline.augment_dataset(dataset)
    \end{python}
\end{frame}

\begin{frame}{Project Plan}
    2. Develop a Python-friendly API that enables:
    \begin{itemize}
        \item Single-method augmentation calls
        \item Composable augmentation pipelines 
        \item<2-> Batch processing capabilities
        \item<3-> Parallel exection options
    \end{itemize}
\end{frame}

\begin{frame}{Project Plan}
    3. Analyse performance against python library tsaug
    \begin{itemize}
        \item Execution time benchmarks
        \item Memory usage comparisons
        \item Quality assessment of augmented data
    \end{itemize}
    
\end{frame}

\begin{frame}
  \frametitle{References}
  \printbibliography[heading=none]
\end{frame}

\end{document}